\documentclass{jarticle}

\usepackage{graphicx}
\usepackage{latexsym}
\usepackage{amsmath}
\usepackage{amssymb}
\usepackage{amsthm}
\usepackage{url}
\usepackage{algorithm}
\usepackage{algorithmicx}
\usepackage{algpseudocode}
\newcommand{\argmin}{\operatornamewithlimits{argmin}}
\newcommand{\dd}{\mathrm{d}}
\newcommand{\ee}{\mathrm{e}}

\renewcommand{\algorithmicrequire}{\textbf{Input:}}

\theoremstyle{definition}
\newtheorem{thm}{定理}
\newtheorem{defi}[thm]{定義}
\newtheorem{prop}[thm]{命題}
\newtheorem{cor}[thm]{系}
\newtheorem{asm}[thm]{仮定}


\title{9 Linear Predictors}
\date{2016年6月3日}

\begin{document}
\maketitle
\begin{abstract}
線形予測:線形分類(halfspaces)、線形回帰(線形関数)、ロジスティック回帰(シグモイド関数 over 線形関数)を紹介する。
\end{abstract}

\section{線形モデル}
\begin{defi}[アフィン関数]
\begin{equation}
L_d = \{ h_{w,b}:w \in \mathbb{R}^d , b \in \mathbb{R} \}, 
\end{equation}
where, 
\begin{equation}
h_{w,b}(x) = \langle w,x \rangle + b = \Sigma_{i=1}^d \omega_ix_i + b.
\end{equation}
\end{defi}
$w' = (b,\omega_1,\omega_2,\omega_3,\dots,\omega_d) ,x'=(1,x_1,x_2,x_3,\dots,x_d)$をすることによってアフィン関数を斉次函数に変換することができる。
\section{線形分類}
\begin{equation}
\begin{aligned}
\mathcal{X} = \mathbb{R} ^d, &\mathcal{Y} =\{-1,+1\} \\
HS_d = sign \circ L_d = \{x \to &sign(h_{w,b}(x)) :h_{w,b} \in L_d\}
\end{aligned}
\end{equation}
\subsection{線型計画法}
\begin{equation}
\begin{aligned}
max_{w \in R^d} \langle u,w \rangle \\
subject \, to \, Aw \ge v
\end{aligned}
\end{equation}
\begin{equation}
\begin{aligned}
max_{w \in R^d} \langle 0,w \rangle \\
subject \, to \, (y_ix_{i,j})w \ge v
\end{aligned}
\end{equation}

\subsection{Perceptron}
\begin{algorithm}
\begin{algorithmic}[1]
           \Require A training set $(x_1,y_1), \dots , (x_m,y_m)$
\end{algorithmic}
\end{algorithm}
\end{document}
